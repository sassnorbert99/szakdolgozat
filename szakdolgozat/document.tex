%
% Szakdolgozat minta az Eszterházy Károly Egyetem
% matematika illetve informatika szakos hallgatóinak.
%

\documentclass[
% opciók nélkül: egyoldalas nyomtatás, elektronikus verzió
% twoside,     % kétoldalas nyomtatás
% tocnopagenum,% oldalszámozás a tartalomjegyzék után kezdődik
]{thesis-ekf}
\usepackage[T1]{fontenc}
\PassOptionsToPackage{defaults=hu-min}{magyar.ldf}
\usepackage[magyar]{babel}
\usepackage{mathtools,amssymb,amsthm}
\usepackage{listings}
\footnotestyle{rule=fourth}

\newtheorem{tetel}{Tétel}[chapter]
\theoremstyle{definition}
\newtheorem{definicio}[tetel]{Definíció}
\theoremstyle{remark}
\newtheorem{megjegyzes}[tetel]{Megjegyzés}

\begin{document}
\institute{Matematikai és Informatikai Intézet}
\title{A szakdolgozat címe}
\author{Szerző neve\\szak}
\supervisor{Dr. Geda Gábor\\beosztás}
\city{Eger}
\date{2021}
\maketitle
\tableofcontents

\chapter*{Bevezetés}
Megújuló energiaforrásnak nevezzük az energiahordozók azon csoportját, amelyek emberi időléptékben képesek megújulni, azaz nem fogynak el, ellentétben a nem megújuló energiaforrásokkal.  A mai civilizáció a zöld energiát helyezi előtérbe, és arra törekszik, hogy minél kisebbre csökkentse az ökológiai lábnyomot. Számos gyakorlati felhasználása van, többek között a villanyautók, tisztán elektromos hajtással működő személygépjárművek, illetve egyéb járművek fejlesztése. Napjainkban számos helyen tapasztaljuk, hogy egyre nagyobb szerepet kap a fenntarthatóság és a környezettudatosság nemcsak a vállalatok és cégek, hanem a fogyasztók gondolkodásában is. Egyre több szerepet kap az életünkben a környezettudatos életmód, a szelektív hulladékgyűjtés és a zöldebb életmód. Számos előnyökkel rendelkeznek a megújuló energiaforrások, például, hogy ezek hosszú távon rendelkezésre álló készletek, szemben a nem megújuló energiaforrásokkal, melyek fosszilis energiahordozók. A fosszilis energiahordozók nem tartanak örökké, hiszen ezeket a földből kinyerve nem lehet őket pótolni, ha már véglegesen elfogytak. Ide tartozik az urán, a kőolaj, földgáz, illetve a szén. Ezen kívül rendelkezik egy másik nagy előnnyel, hogy működésük rendkívül környezetkímélő. A fosszilis energiahordozók égetése hatalmas mennyiségű szén-dioxidot bocsát ki, ezzel mesterségesen növelve az üvegházhatás folyamatát a Földünkön, ezzel szemben a megújuló energiaforrások használatával sokkal kevesebb károsanyag kerül a légkörbe, melyeknek felhasználását egyre több ország helyezi előtérbe, hogy ezzel is mérsékelni tudják a globális felmelegedés problémáját.
\par Szakdolgozatunkban  a megújuló energiaforrások tudatos, és környezetvédő felhasználását szeretnénk modellezni.  A modellünkben az energiaforrásokat mikrokontrollerrel 
ötvözve szeretnénk a leghatékonyabban szabályozni intelligens módon, azaz a rendszer képes önállóan optimalizálni a termelt és a felhasznált energia mennyiségét. Gyakorlati felhasználásban terepasztalon elhelyezett kisméretű modelleken szemléltetjük a különféle erőműveket, illetve energiatároló technológiákat, fogyasztókat. Fogyasztóinknak gyakorlati felhasználásuk lesz, mely azt jelenti, hogy a való életben megtalálható általános fogyasztókkal fogjuk szimulálni a projektet. Ilyenek lehetnek a családi házak, háztömbök, elektromos töltőállomások, iskolák és gyárak.

\chapter{Tudományos Diákköri Konferencia}
	\section{Szakasz címe}
		\subsection{Alszakasz címe}
		Lórum ipse olyan borzasztóan cogális patás, ami fogás nélkül nem varkál megfelelően. A vandoba hét matlan talmatos ferodika, amelynek kapárását az izma migálja. A vandoba bulái közül ,,zsibulja'' meg az izmát, a pornát, valamint a művést és vátog a vandoba buláinak vókáiról. Vókája a raktil prozása két emen között. Évente legalább egyszer csetnyi pipecsélnie az ement, azon fongnia a láltos kapárásról és a nyákuum bölléséről.
		
		A vandoba ninti és az emen elé redőzi a szamlan radalmakan érvést. Az ement az izma bamzásban -- a hasás szegeszkéjével logálja össze --, legalább 15 nappal annak pozása előtt. Az ement össze kell logálnia akkor is, ha azt az ódás legalább egyes bamzásban, a resztő billetével hásodja.

\chapter{Rendszerről összefoglalóan}
	\section{Szakasz címe}
		\subsection{Előnyök és hátrányok}
		Lórum ipse olyan borzasztóan cogális patás, ami fogás nélkül nem varkál megfelelően. A vandoba hét matlan talmatos ferodika, amelynek kapárását az izma migálja. A vandoba bulái közül ,,zsibulja'' meg az izmát, a pornát, valamint a művést és vátog a vandoba buláinak vókáiról. Vókája a raktil prozása két emen között. Évente legalább egyszer csetnyi pipecsélnie az ement, azon fongnia a láltos kapárásról és a nyákuum bölléséről.
		
		A vandoba ninti és az emen elé redőzi a szamlan radalmakan érvést. Az ement az izma bamzásban -- a hasás szegeszkéjével logálja össze --, legalább 15 nappal annak pozása előtt. Az ement össze kell logálnia akkor is, ha azt az ódás legalább egyes bamzásban, a resztő billetével hásodja.
	\section{rajzok ismertetése}

\chapter{Rendszerünk működése}
	\section{Terepasztalunk működése}
		\subsection{Napállás kalkulátor}
			\textbf{A feladat elkészítéséért Sass-Gyarmati Norbert a felelős.} 
			\par Feladatom ebben a cikkben az volt, hogy matematikai úton megvalósítsak egy szoftvert, mely egy adott napból és egy szélességi, hosszúsági fokból képes legyen kiszámolni az adott nap deklinációs szögét, azaz a Nap szöghelyzetét a szoláris délben. Ezen kívül kitudja számolni a rendszer az adott nap naphosszát, a napfelkelte és naplemente megközelítő értékét, zenit szöget, illetve abból a Napmagassági szöget. Továbbá képes megmondani, hogy a terepasztalunk hány fokos szögben kell elforduljon ahhoz, hogy az arányokat megtartva szimulálhassunk egy ország éghajlati és napszaki adatait. További feladata, hogy képes megmondani, hogy egy fixen telepített déli tájolású 40 fokos naprendszer különböző országokban eltérő napokban milyen szögbe esik, illetve az ebbe eső szögnek milyen termelői hatékonysága van.
			\par Feladatomat Python nyelven írtam, melyhez a PyCharm Community Edition keretrendszert használtam. 
			\begin{definicio}
				Deklináció: a Nap szöghelyzete a szoláris délben (azaz ha a Nap a helyi délkörön van) az egyenlítő síkjához viszonyítva. A Föld a Nap körül ellipszis pályán kering, miközben maga a Föld is forog saját tengelye körül. A földpálya síkja és az Egyenlítő által meghatározott sík egymással szöget zár be, azaz a Föld forgásának a tengelye szöget zár be a földpálya síkjára állított merőlegessel. Értéke a napközeli és a naptávoli pontban 23,5, a tavaszi és az őszi napéjegyenlőség idején zérus. Északon pozitív. –23,5 <=  Deklináció <= 23,5. A deklinációs szög értelmezését az 1.5. ábra mutatja.
				\par KÉP!!!!!!!
				\cite{Kornyezet}
			\end{definicio}
			\begin{definicio}
				Nappalhossza: meghatározható a napkelte óraszögéből. Mivel a napkeltétől a delelésig ugyanannyi idő telik el, mint a deleléstől napnyugtáig, a nappal hossza 2ws lesz, ezt elosztva 15 fokkal megkapjuk a nappal hosszát órában.
				\[Nd = (2/15) ws\]
				\cite{Kornyezet}
			\end{definicio}
			\begin{definicio}
				Zenitszög: a függőleges és a Naphoz húzott egyenes által bezárt szög, azaz a vízszintes felületre érkező sugárzás beesési szöge. Adott időben a megfigyelőnek a Földön meghatározható a pozíciója, ezt nevezzük a megfigyelő zenitjének. Ez a pont metszéspontja a megfigyelő helyének földfelszíni normálisának és az égi mezőnek. Az ezzel a ponttal ellentétes helyen lévő zenitet nevezzük nadírnak. A megfigyelő horizontja egy nagy kör (az égi mezőben), egy olyan sík, amely átmegy a Föld középpontján és amelynek határát a zenit és a Föld normálisának metszővonala jelenti. A zenit szög az a szög, amely a lokális zenit, valamint a Nap és a megfigyelő által meghatározott egyenes egymással bezár. Ez a szög 0 és 90 fok között változhat.
				\cite{Kornyezet}
			\end{definicio}
			\begin{definicio}
				Napmagasság szöge: a Napnak szögben kifejezett magassága a megfigyelő horizontjából, azaz a vízszintes és a Naphoz húzott egyenes által bezárt szög. Értéke 0 és 90 fok között van. A napmagasság szöge komplementere a zenit szögnek.
				\cite{Kornyezet}
			\end{definicio}
			\par Ezek definíciók alapján sikerült egy szélességi és hosszúsági fok, illetve egy a felhasználó által kiválasztott dátum megadásával kiszámolni az adott ország deklinációs, zenit, Napmagassági szög, illetve nappalhosszát kiszámolni.
			\par KÉP A PYTHONNAL KÉSZÍTETT KÓDRÉSZLETRŐL!!!!!!!!!!!!!
			\par A további adatok kiszámításához (napfelkelte, naplemente, asztal elfordulása) ezek adatok elengedhetetlenek.
			\subsubsection{Napfelkelte, naplemente kiszámítása}
				\par A napfelkelte és napnyugta meghatározása a pontos földrajzi koordináták (hosszúsági és szélességi fok) valamint a dátum alapján történik.\cite{kiszamolo}
				\par Kiszámításához egy metódust hoztam létre, melnyek a  ,,calcsunriseandsunset" nevet adtam. Ez a metódus egy dátumot vár a felhasználótól, illetve a rendszer felhasználja az általa eltárolt latitude és longitude fokokat. A dátumból ezután egy julián dátumot készítünk, mely a Kr. e. 4713. év első napjától eltelt napok számával és óra-perc-másodperc helyett a nap decimális törtrészeivel adja meg az időpontokat.\cite{julian}
				\par Ezután csinálunk egy julián csillagot, melyhez már a hosszúsági fokot is felhasználjuk. Matematikai képletek segítségével pedig ezek segítségével kiszámoljuk a napfelkelte és naplemente értékét
				\lstinputlisting[language=Python]{calcriseset.py}
			\subsubsection{Asztal forgatása}
				\par Az asztal forgatásához a deklinációs\cite{Kornyezet} szöget vettem segítségül. A deklinációs szög megmondja, hogy egy adott napban egy országban a Nap milyen szögben helyezkedik el szoláris délben, így képes minden ország beállításával is megmondani ezt a bizonyos szöget. Ehhez viszonyítottam az asztal eltolását, hiszen egy Magyarországhoz északibb ország különböző hónapokban más viselkedést produkálnak, például nyáron hegyesebb, télen tompább szög. Ugyanez igaz a nálunk délibb országokhoz is, csak ott ezek negáltja történik. 
				\par Ez a bizonyos szög országon belül is havonta változik, minden nap maximum 0,5 fokkal. Az asztal tehát a deklinációs szög negáltjával fog változni, hiszen az arányokat megtartván az északi pontot helyezzük a megváltozott érték helyére.
				\lstinputlisting[language=Python]{table.py}
				\par A napcellák elhelyezkedése a Naphoz képest, illetve a pontos szög meghatározása ugyanebben a metódusban definiált. Az égtájak konstans értékként vannak definiálva, melyben az értékek fokban értendők. A napcellák elhelyezkedése tehát a Dél - deklináció. A napcellák hatékonyságáról\cite{gershoj} tanulmányok szerint a déli 35-40 fok az optimális, így a rendszert is úgy állítottuk be, hogy egy magyarországi déli tájolású 40 fokban legyen.
				\lstinputlisting[language=Python]{solar.py}
				\par FOLYTATÁSSSSS!!!!!!!
			\subsubsection{Hibakezelések}
				Lorem ipsum
		\subsection{Napállás kalkulátor - Rest API}
			A feladat elkészítéséért Oravecz Zsolt a felelős. 
	\section{Szimulációk}
		\subsection{Fogyasztó probléma szimulálása}
			\par \textbf{A feladat elkészítéséért Sass-Gyarmati Norbert a felelős}
			\par Ahhoz, hogy a fogyasztók energiaigényeiket feltudjuk mérni, szükségünk volt egy szimulátor programra, melyben megtudjuk adni, hogy menni fogyasztó fogja használni a terepasztalunkat. A fogyasztók az alábbiak lehetnek:
			\begin{itemize}
				\item Családi házak
				\item Lakások
				\item Iskolák
				\item Kórházak
			\end{itemize}
			\par A fogyasztók különböző energiaigényekkel rendelkeznek. Egy átlag családi háznak (4 fős) nagyjából 230 kWh energiára van szüksége egy hónapban. Kutatásaink során további statisztikát tudtunk levonni, amiben kiderült, hogy egy lakásnak, vagy bérháznak (4 fős) nagyjából 200 kWh energiára van szüksége\cite{kWh}. Az iskolákat, illetve kórházakat más matematikai műveletekkel lehet megadni. 
			\par Először szükségünk volt olyan adatokra, hogy átlagosan mennyi energiát fogyasztanak ezek a különféle intézmények. A méréseket egy 2001-es tanulmány alapján készítettem, így minden érték csak egy megközelítő értéket ad. Mérések alapján kiderült, hogy egy átlagos iskolának nagyjából 20 kWh/$m^{2}$ energiaigényre van szüksége, míg egy kórháznak jelentősen nagyobb, körülbelül 100 kWh/$m^{2}$ energiaigényre van szüksége\cite{school}. 
			\par Ezek után szükségem volt egy szabványra mely kimondja, hogy egy átlagos iskola, illetve kórház milyen szabályoknak kell megfeleljenek. A kutatásaimat felhasználva egy iskola hivatalos szabványa, hogy 2,5$m^{2}$ jut egy diák számára\cite{school_m2}. Tehát egy iskola kalkulálása ezek szorzatából tevődik össze. Továbbá kutatásaim arra is rámutattak, hogy egy kórházban 6-8$m^{2}$ jut egy beteg részére. Így az iskola, illetve a kórház mérete és ebből kiszámítva a fogyasztása nagyban függ az intézményekben tartózkodó tanulók, vagy betegektől.
			\par Szimulátor programom Python nyelven írtam a PyCharm\footnote{A PyCharm egy integrált fejlesztői környezet (IDE), amelyet a számítógépes programozásban használnak, kifejezetten a Python nyelv számára\cite{pycharm}.} segítségével. A szimulátorban a felhasználó konzolos ablakon keresztül megadhatja a szoftvernek a házak, lakások, iskolák, kórházak, tanulók, illetve betegek számát, majd ezek adatokból képes kiszámolni az átlagos fogyasztási igényt, valamint ugyanezt az adatot lebontja napra pontosan.
			\par Szoftveremben minden intézmény egy külön osztály, melyek más-más számításokat kell végezzenek az igény kiszámításához. 
			\lstinputlisting[language=Python]{consumers.py}
			\par Ahogy a kódrészletből is látható, egy családi ház konstruktorának fogyasztási értéke 230, míg egy lakás átlagos fogyasztási értéke 200. Napra lebontva pedig egy átlagos naphosszt választottam, így minden értéket 30-al oszt el, így lebontván napi szintre az értékeket. 
			\par Az iskolák és kórházak összetettebb konstruktorral rendelkeznek, melyek nagyban függnek a paraméterként megadott tanulók, illetve betegektől.
			\lstinputlisting[language=Python]{calculate_estates.py}	
			\par Ahogy a kódrészletből tisztán látható, az iskola és kórház konstruktorai már egy tanuló, illetve beteg paramétert is várnak, melyből kitudja számolni az intézmény átlagos négyzetméterét, illetve ebből az adatból képes kiszámolni a négyzetméterre jutó energia igényüket.
			\par A különféle fogyasztók értékeit listában tárolom, illetve ciklusok segítségével tudom feltölteni. A felhasználó a program lefutása után konzolból megadhatja a különböző értékeket, mely során a szoftver egyes osztályait meghívván, beállítja a számára szükséges értékekkel a paramétereit, majd közli a felhasználóval az alábbi adatokat:
			\lstinputlisting[language=Python]{estate.py}	
							
			
			
			
			
			
		\subsection{Vízerőmű szimulálása}
		\subsection{Napcellák teljesítményének szimulálása}
		\subsection{Encoder működésének szimulálása}
	\section{optimalizálás}
		\subsection{telepített napcellák optimalizálása, tájolása}
	\section{Termelők és fogyasztók}
		\subsection{termelők ismertetése}
		\subsection{fogyasztók ismertetése}
		\par A fényforrások, és egyéb elektronikai eszközök a különböző energia felhasználású fogyasztókat fogják modellezni. Célunk valósághűen modellezni a fogyasztókat. (Kisméretű házak, épületek.) 
		\par Az alábbi modellek lesznek a terepasztalunkon:
		\begin{itemize}
			\item Családi házak (átlag 4 fős, fogyasztása körülbelül 230 kWh/hó)
			\item Bérházak (átlag 4 fős, fogyasztása körülbelül 200 kWh/hó)
			\item Tömbházak (bérházak fogyasztásától függően változik)
			\item Elektromos töltőállomások (használattól függően változik)
			\item Közvilágítás (alkalmazástól függően változik)
		\end{itemize}
	
		Modellünk olyan fogyasztási értékeket fog szemléltetni, mely a valóságnak arányosan eleget tesz. A fogyasztók számát dinamikusan lehet majd szabályozni, mely hatással lesz a rendszer működésére. A modellünkben a fogyasztók különböző nyitófeszültségű LED-ek lesznek, melyekkel a fogyasztók energiafelhasználását tudjuk szimbolizálni. A projektünkben a fogyasztók egységes áramot használnak, azonban a számítások során a valóságnak megfelelő értékekkel számolunk.
		
	\section{modellek}
		\subsection{Időjárás állomás}
			\par Modellünk tartalmaz egy kis éghajlat elemző műszert is, melyre egy 16x2-es LCD kijelző van csatolva, amin adatokat tudunk leolvasni az éppen aktuális hőmérsékletről és páratartalomról. Ez a műszer szemlélteti modellünk aktuális éghajlati adottságait.
		\subsection{Fogyasztók modellezése}
			\par A fényforrások és egyéb elektronikai eszközök a különböző energia felhasználású fogyasztókat modellezik. Célunk valósághűen modellezni a fogyasztókat. (Kisméretű házak, épületek, gyárak, iskolák, melyek más fogyasztási igényekkel vannak ellátva)
			
	\section{prototípusok}
		\subsection{telepített napcellák prototípusai}
		\subsection{intelligens napcellák prototípusai}
 	\section{Napcellák}
 		\subsection{telepített napcellák}
 		\subsection{intelligens napcellák}
 		\subsection{napcellák integrációja}
 	\section{Vízerőmű}
 	
 	

	
	
		

\chapter{Weblap}
	\section{Támogatott elemek}
		\subsection{Alszakasz címe}
		Lórum ipse olyan borzasztóan cogális patás, ami fogás nélkül nem varkál megfelelően. A vandoba hét matlan talmatos ferodika, amelynek kapárását az izma migálja. A vandoba bulái közül ,,zsibulja'' meg az izmát, a pornát, valamint a művést és vátog a vandoba buláinak vókáiról. Vókája a raktil prozása két emen között. Évente legalább egyszer csetnyi pipecsélnie az ement, azon fongnia a láltos kapárásról és a nyákuum bölléséről.
		
		A vandoba ninti és az emen elé redőzi a szamlan radalmakan érvést. Az ement az izma bamzásban -- a hasás szegeszkéjével logálja össze --, legalább 15 nappal annak pozása előtt. Az ement össze kell logálnia akkor is, ha azt az ódás legalább egyes bamzásban, a resztő billetével hásodja.
	\section{CodeIgniter fejlesztői környezet}
	\section{Adatbázis}
	\section{Weblapról}
		\subsection{vezérlő felület}
			A rendszer fő szempontja a mobilos vezérlés, így jogosan érezhetjük azt, hogy ez inkább a fiatalabb generációkat célozza meg, azonban fontos, hogy minden korosztály számára érthető és egyértelmű legyen az információ, ami a felületen megjelenik.
			Első lépésként a látogatóknak regisztrálni kell a felület használatához. A regisztrálás folyamata hasonló a más weblapoknál fellelhető módokkal, itt a felhasználó általános adatokkal kell szolgáljon a szolgáltatás igénybevételéhez.
			
			
			\par KÉP!!!!!!!!!!
			
			\par Ahogy a képeken is látható, a felhasználónak rendelkeznie kell egy teljes névvel, irányítószámmal, email címmel, felhasználónévvel, valamint egy jelszóval. Az első képen a felhasználó számítógépes felületről tudja elérni, míg a második kép már telefonos felületen elérhető.
			\par Természetesen a regisztrált felhasználóknak a bejelentkezés gombra kattintva egyből a kezdőlapra tud bejutni.
			
			
			
		\subsection{Beléptető modul}
		\subsection{Kezdőlap}
			A kezdőlapon a varázstorony aktuális hírei érhetők el, e-mail címek és nyitvatartási rendek. Kezdőlapunk egy már meglévő weblapnak alapját dolgozza fel
			\par \url{(https://uni-eszterhazy.hu/hu/egyetem/kultura/varazstorony)}.
		\subsection{Rólunk}
			\par A fejléc következő része a Rólunk ablak, melyben a projektben résztvevő fejlesztők és egyéb szerkesztők neveit olvashatjuk. Ez az ablak ismerteti a felhasználókkal az egyes modulok felelőseit, forrásait.
		\subsection{Blog}
			\par A blog oldal azért készült, hogy a felhasználók észrevételeket, tapasztalatokat és egyéb véleményeket tudjanak feltölteni, ezáltal egymással is tudnak kommunikálni. A blogban lehet képet is feltölteni, valamint egyes kategóriák által lehet csoportosítani. A kategóriák a varázstoronyban megtalálható eszközök. További kategóriák létrehozásához admin szintű felhasználóra van szükség. Amennyiben igény keletkezik egy új kategória létrehozásához, úgy a felhasználók írhatnak a rendszer admin szintű felhasználóinak, ami átvizsgálás után létre is jön. A blogban továbbá lehet írni egy részletes leírást a témáról. Egy blog küldése után a rendszer megjegyzi az küldés utáni naptári időpontot, melyet a leírás fölött kiír.
			\par KÉP!!!!!!!!!!!
			\par Ahogyan a képeken is láthatjuk, weblapunk első posztja a Projekt1 kategóriába tartozik, ahol egy 16x2-es lcd kijelzőről készült képet is feltöltöttünk. Fontos azt is megjegyezni, hogy kategóriák azért kellenek, hogy később a posztokat listázni tudjuk kategóriák segítségével. Ha egy felhasználó csak egy bizonyos kategória iránt érdeklődik, lehetősége van azokat kilistázni, ezáltal egy kényelmesebb és könnyen kezelhető felület tárul elé. Weblapunk nagy hangsúlyt fektet a felhasználóbarát webes megjelenítésre, így egy letisztult és kényelmes weblap jelenik meg minden felhasználóink számára.
		\subsection{Kategóriák}
			\par Mint már említettük, szoftverünk tartalmaz egy kategória ablakot, melyben az eddig feltöltött összes kategória közül tud választani a felhasználó. Egy szabadon választott kategória kattintásra kilistázza az eddigi összes olyan posztot, észrevételeket és egyéb tartalmakat, melyek abban a kategóriában szerepelnek.  Ezáltal a felhasználó csak azokat a kategóriában szereplő tartalmakat olvashatja, amelyek érdeklik. A kategóriák a varázstoronyban szereplő eszközök, melyeket admin szintű felhasználók, illetve rendszer karbantartók tudnak módosítani, mezei felhasználónknak azonban személyes igény esetén lehetőségük van írni az üzemeltetőknek.
			\par KÉP!!!!!!!!!!
			\par Ahogy a képeken látható, weblapunk létrehozása után két kategóriát töltöttünk fel, melynek kattintására a kategória által létrehozott posztot olvashatjuk. Míg az első ábrán számítógépes felületről nyitottuk meg, a felhasználók számára kényelmesebb, hiszen a fejlécben minden információt láthatnak. A második ábra telefonról készült, így a telefonos megjelenítés szempontjából a fejléc tartalmait elrejtettük, mely a bal felső ikon kattintására kilistázódik. Szoftverünk multi platformos, tesztelve lett Windows-on, Linuxon, illetve MacOS alatt. Telefonon tesztelve lett Android, illetve IOS készülékeken.
			\par További előnyként szolgál az is, hogy a kategóriák ABC sorrendben listázódnak ki, ezáltal további könnyedséggel szolgál egyes kategóriákat elérése.
		\subsection{Térkép}
			\par A felület segítségével a felhasználók idegenvezető nélkül bejárhatják a Varázstorony termeit, és különböző leírások segítik az egyes eszközök megismerését. Célunk, hogy azok a felhasználók, akik még nem jártak a varázstoronyban, tudjanak tájékozódni és ki tudják keresni a számukra érdekes témákat, melyről rendszerünk képekkel, információkkal és egyéb interaktív dolgokkal szolgál. A térkép fülre kattintva a varázstorony szintenkénti alaprajza található, ahol minden terem, folyosó, ahol eszközök találhatók, fel van tüntetve. Három fajta feltüntetés van a rendszerünkben.
			\par
			\begin{enumerate}
				\item Megtekinthető tartalom:
				\begin{itemize}
					\item Felhasználóink meg tudják webes felületről tekinteni az egyes termek érdekességeit. A gombra kattintva egy pop-up szerű kép jelenik meg az egyes eszközökről.
				\end{itemize}
				\item Interaktív tartalom:
				\begin{itemize}
					\item Felhasználóink számára biztosítunk interaktív vetélkedőket egyes eszközök kattintása után. Ezek lehetnek kvízek, csoportos mini feladatok. 
				\end{itemize}
				\item Vezérelhető tartalom:
				\begin{itemize}
					\item Felhasználónk ilyen típusú gombra kattintva az olvasás és a megjelenő kép mellett vezérelni is tudja egyes eszközöket.
				\end{itemize}
			\end{enumerate}
		\subsection{Jelmagyarázat a térképhez}
			szoftverünk könnyebb értelmezése érdekében létrehoztunk egy jelmagyarázatot, melyben az egyes tartalmak funkcióit tároljuk. Weblapunk három funkciót biztosít a felhasználók számára:
			\begin{itemize}
				\item megtekinthető
				\item interaktív
				\item vezérelhető
			\end{itemize}
			\par A funkciók mellé szín is társul.
			\par KÉP !!!!!!!!!!!!!!!!!!!!!!!!!!!!!!!!!!!!!!!!!!!!!
			\par Ahogy a mellékelt képen is láthatjuk, a megtekinthető tartalmak színe piros, azok a tartalmak, melyek interaktív feladatokat tartalmaznak sárgák, végül a tartalmak, melyeket vezérelni is lehet, kékes zöld színűek.
			
		\subsection{Eszközök}
			\par A felhasználóknak lehetőségük van egyes eszközöket részletesebben tanulmányozni, mely az eszközök ablakra kattintva lesz elérhető. A gombra kattintva eléjük tárul az általunk fejlesztett projektek részletes beszámolója, illetve azok leírása, egyéb tartalma. Ezek természetesen a térkép menüpont alatt is megtalálhatók, hiszen azok gombaira kattintva átirányítja felhasználóinkat az általuk választott oldalra.
			\par KÉP !!!!!!!!!!!!!!!!!!!!!!!!!!!!!!!!!!!!!!!!!!!!!
			\par Az első képen a Cartesius-búvár, illetve annak részletes leírása található, míg a második képen a terepasztal, mely egy intelligensen működő energetikai rendszert valósít meg.
			
		\subsection{Felhasználók}
			\par Weblapunk rendelkezik admin szintű felhasználókkal, melyek feladata a kategóriák, illetve egyes posztok karbantartása. Így az admin felhasználók fejléce kiegészül egy “Kategória készítése” menüponttal, melyben az általa, vagy közösen megbeszélt kategóriákat tudja feltölteni. Adminként nem lehet regisztrálni, ezt a rendszer tulajdonostól lehet igényelni, melyet a rendszer karbantartó át ír az adatbázison keresztül.
			\par KÉP !!!!!!!!!!!!!!!!!!!!!!!!!!!!!!!!!!!!!!!!!!
			\par Ahogyan a képen is látható, az admin továbbá rendelkezik egy Users menüponttal, melyben megtekintheti az egyes usereket (regisztrált felhasználókat), illetve azok adatait adatbiztonság céljából. Továbbá megtekintheti, hogy kik adminok a rendszer felhasználói közül.
			\par KÉP !!!!!!!!!!!!!!!!!!!!!!!!!!!!!!!!!!!!!!!!!!
			\par A képen látható adatokat szándékosan nem jelenítjük meg adatbiztonság érdekében. Ahogy az ábra is mutatja, listázva vannak a felhasználók. Az admin szintnek két lehetséges értéke van, 0, ha a felhasználó nem admin, 1, ha a felhasználó admin.
			\par Ahogy a képen láthatjuk, az 1-es, 4-es és 5-ös ID-vel rendelkező felhasználóink admin szintje 1, tehát admin szintű felhasználó.
		\subsection{Poszt készítése}
		\subsection{Kategória készítése}
		

\chapter{Fejlesztői környezetek és publikációi}
	\par Ebben a fejezetben bemutatjuk az általunk használt fejlesztői környezetet és a fejlesztéshez szükséges komponenseket, azok tulajdonságait, illetve funkcióit.
	\section{Git verziókövető rendszer}
	Mivel ketten dolgoztunk a terepasztal projekten, meg kellett oldanunk, hogy szimultán tudjunk dolgozni, azaz egymástól függetlenül. A projekt első verzióit egy tárhelyre töltöttük fel, melyről mindig le kellett tölteni az aktuális verziót, majd vissza feltölteni az új, módosítottat. Ezzel a módszerrel egyszerre csak egy ember tudott dolgozni, ami nagyon megnahezítette a fejlesztési tevékenységünket.
	\par Szükségünk volt egy verziókövető rendszer elsajátításához. Ezen rendszerek legnagyobb előnyei, hogy egy projekten többen is dolgozhatunk egyszerre, anélkül, hogy egymás munkáját hátráltatnánk, illetve ha valaki változtatást készít és feltölti, azt a rendszer nyomon tudja követni. Ha ketten egyszerre ugyanazon az állományon végeznek módosítást, a rendszer feltöltéskor megpróbálja összefésülni (merge) a módosításokat, ha nem sikerül, jól láthatóan megjeleníti az ütközéseket. Ilyen esetekben megtudjuk nézni a konfliktust okozó állományokat és lehetőségünk van a két állományt manuálisan összefésülni. Ezzel a módszerrel folyamatosan szinkronban lehetett mindkettőnk munkája.
	\par A Git verziókövető rendszert választottuk, mert korábban már használtuk Windows, illetve Linux rendszeren, és jó tapasztalataink vannak róla. Korábbi kurzusainkon is használtuk, így könnyebb volt a Git verziókövetőt elsajátítani. 
	\par Ahhoz, hogy fejlesztés közben ne hátráltassuk egymás munkáját, szükség van egy kliensre, mely könnyen kezelhető felületet biztosít a hozzáféréshez, a projekt klónozásához, feltöltéshez, stb. Windows alatt a Github Desktopot használjuk, MacOS alatt pedig terminálban kezeljük a verziókövető funkcióit.
	\section{Trello feladatkövető rendszer}
	\par Ahhoz, hogy feltudjuk osztani kettőnk közt a feladatokat, szükségünk volt a verziókövető rendszeren kívül egy feladatkövető rendszerhez is. A projekt megkezdése előtt a feladatokat szóban, illetve papíralapon osztottuk fel, azonban egy idő után átláthatatlanná vált a feladatok megosztása. Szükségünk volt egy feladatkövető rendszer elsajátításában. Ezen rendszerek legnagyobb előnye, hogy táblázatokban tudjuk összefoglalni a feladatokat, illetve azokhoz könnyen hozzátudjuk rendelni a fejlesztőket. Választásunk a Trello feladatkövető rendszerre esett, mert korábban már használtuk, így könnyebb volt a rendszer elsajátítása. 
	
	\section{Technológiák}
		\subsection{Python}
		Lórum ipse olyan borzasztóan cogális patás, ami fogás nélkül nem varkál megfelelően. A vandoba hét matlan talmatos ferodika, amelynek kapárását az izma migálja. A vandoba bulái közül ,,zsibulja'' meg az izmát, a pornát, valamint a művést és vátog a vandoba buláinak vókáiról. Vókája a raktil prozása két emen között. Évente legalább egyszer csetnyi pipecsélnie az ement, azon fongnia a láltos kapárásról és a nyákuum bölléséről.
		
		A vandoba ninti és az emen elé redőzi a szamlan radalmakan érvést. Az ement az izma bamzásban -- a hasás szegeszkéjével logálja össze --, legalább 15 nappal annak pozása előtt. Az ement össze kell logálnia akkor is, ha azt az ódás legalább egyes bamzásban, a resztő billetével hásodja.
		\subsection{Python és a C nyelv integrációja}
		Lórum ipse olyan borzasztóan cogális patás, ami fogás nélkül nem varkál megfelelően. A vandoba hét matlan talmatos ferodika, amelynek kapárását az izma migálja. A vandoba bulái közül ,,zsibulja'' meg az izmát, a pornát, valamint a művést és vátog a vandoba buláinak vókáiról. Vókája a raktil prozása két emen között. Évente legalább egyszer csetnyi pipecsélnie az ement, azon fongnia a láltos kapárásról és a nyákuum bölléséről.
		
		A vandoba ninti és az emen elé redőzi a szamlan radalmakan érvést. Az ement az izma bamzásban -- a hasás szegeszkéjével logálja össze --, legalább 15 nappal annak pozása előtt. Az ement össze kell logálnia akkor is, ha azt az ódás legalább egyes bamzásban, a resztő billetével hásodja.
		\subsection{PHP nyelv}
	\section{Arduino}
		\subsection{szenzorok és kellékek ismertetése}
		


\begin{tetel}
Tétel szövege.
\end{tetel}

\begin{proof}
Bizonyítás szövege.
\end{proof}

\begin{definicio}
Definíció szövege.
\end{definicio}

\begin{megjegyzes}
Megjegyzés szövege.
\end{megjegyzes}

\begin{thebibliography}{2}
\bibitem{kiszamolo}
\textsc{\url{https://kiszamolo.com/napfelkelte-napnyugta-kalkulator/}}.
\bibitem{julian}
\textsc{\url{https://hu.wikipedia.org/wiki/Julián_dátum}}
\bibitem{Kornyezet}
\textsc{dr. Barótfi István}:  \emph{Környezettechnika}
\bibitem{gershoj}
\textsc{https://gershojenergia.com/napelem-kisokos/optimalis-napelem-elhelyezes/}
\bibitem{kWh}
\textsc{\url{https://elmuemasz.hu/egyetemes-szolgaltatas/szolgaltatasok/villamos-energia/aramdij-kalkulatorok/lakossagi-aramdij-elmu?fbclid=IwAR3UKUacdAXQeBuYPsDckohKhAf_hY2TnitQqGmBwRiFq_ye-o84onVUPrc}}
\bibitem{school}
\textsc{\url{http://www.personal.ceu.hu/students/03/Alexandra_Novikova/2/El\%20tertiary\%20site\%20folders/documents/description_of_eltertiary_for_schools_hu6_ver2.pdf}}
\bibitem{school_m2}
\textsc{\url{https://www.origo.hu/itthon/20010414szabvany.html}}
\bibitem{pycharm}
\textsc{https://en.wikipedia.org/wiki/PyCharm}

\end{thebibliography}
\end{document}