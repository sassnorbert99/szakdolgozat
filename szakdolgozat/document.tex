%
% Szakdolgozat minta az Eszterházy Károly Egyetem
% matematika illetve informatika szakos hallgatóinak.
%

\documentclass[
% opciók nélkül: egyoldalas nyomtatás, elektronikus verzió
% twoside,     % kétoldalas nyomtatás
% tocnopagenum,% oldalszámozás a tartalomjegyzék után kezdődik
]{thesis-ekf}
\usepackage[T1]{fontenc}
\PassOptionsToPackage{defaults=hu-min}{magyar.ldf}
\usepackage[magyar]{babel}
\usepackage{mathtools,amssymb,amsthm}
\footnotestyle{rule=fourth}

\newtheorem{tetel}{Tétel}[chapter]
\theoremstyle{definition}
\newtheorem{definicio}[tetel]{Definíció}
\theoremstyle{remark}
\newtheorem{megjegyzes}[tetel]{Megjegyzés}

\begin{document}
\institute{Matematikai és Informatikai Intézet}
\title{A szakdolgozat címe}
\author{Szerző neve\\szak}
\supervisor{Tanár neve\\beosztás}
\city{Eger}
\date{2020}
\maketitle
\tableofcontents

\chapter*{Bevezetés}
Megújuló energiaforrásnak nevezzük az energiahordozók azon csoportját, amelyek emberi időléptékben képesek megújulni, azaz nem fogynak el, ellentétben a nem megújuló energiaforrásokkal.  A mai civilizáció a zöld energiát helyezi előtérbe, és arra törekszik, hogy minél kisebbre csökkentse az ökológiai lábnyomot. Számos gyakorlati felhasználása van, többek között a villanyautók, tisztán elektromos hajtással működő személygépjárművek, illetve egyéb járművek fejlesztése. Napjainkban számos helyen tapasztaljuk, hogy egyre nagyobb szerepet kap a fenntarthatóság és a környezettudatosság nemcsak a vállalatok és cégek, hanem a fogyasztók gondolkodásában is. Egyre több szerepet kap az életünkben a környezettudatos életmód, a szelektív hulladékgyűjtés és a zöldebb életmód. Számos előnyökkel rendelkeznek a megújuló energiaforrások, például, hogy ezek hosszú távon rendelkezésre álló készletek, szemben a nem megújuló energiaforrásokkal, melyek fosszilis energiahordozók. A fosszilis energiahordozók nem tartanak örökké, hiszen ezeket a földből kinyerve nem lehet őket pótolni, ha már véglegesen elfogytak. Ide tartozik az urán, a kőolaj, földgáz, illetve a szén. Ezen kívül rendelkezik egy másik nagy előnnyel, hogy működésük rendkívül környezetkímélő. A fosszilis energiahordozók égetése hatalmas mennyiségű szén-dioxidot bocsát ki, ezzel mesterségesen növelve az üvegházhatás folyamatát a Földünkön, ezzel szemben a megújuló energiaforrások használatával sokkal kevesebb károsanyag kerül a légkörbe, melyeknek felhasználását egyre több ország helyezi előtérbe, hogy ezzel is mérsékelni tudják a globális felmelegedés problémáját.
\par Szakdolgozatunkban  a megújuló energiaforrások tudatos, és környezetvédő felhasználását szeretnénk modellezni.  A modellünkben az energiaforrásokat mikrokontrollerrel 
ötvözve szeretnénk a leghatékonyabban szabályozni intelligens módon, azaz a rendszer képes önállóan optimalizálni a termelt és a felhasznált energia mennyiségét. Gyakorlati felhasználásban terepasztalon elhelyezett kisméretű modelleken szemléltetjük a különféle erőműveket, illetve energiatároló technológiákat, fogyasztókat. Fogyasztóinknak gyakorlati felhasználásuk lesz, mely azt jelenti, hogy a való életben megtalálható általános fogyasztókkal fogjuk szimulálni a projektet. Ilyenek lehetnek a családi házak, háztömbök, elektromos töltőállomások, iskolák és gyárak.

\chapter{Tudományos Diákköri Konferencia}
	\section{Szakasz címe}
		\subsection{Alszakasz címe}
		Lórum ipse olyan borzasztóan cogális patás, ami fogás nélkül nem varkál megfelelően. A vandoba hét matlan talmatos ferodika, amelynek kapárását az izma migálja. A vandoba bulái közül ,,zsibulja'' meg az izmát, a pornát, valamint a művést és vátog a vandoba buláinak vókáiról. Vókája a raktil prozása két emen között. Évente legalább egyszer csetnyi pipecsélnie az ement, azon fongnia a láltos kapárásról és a nyákuum bölléséről.
		\cite[102.~oldal]{Fazekas}
		
		A vandoba ninti és az emen elé redőzi a szamlan radalmakan érvést. Az ement az izma bamzásban -- a hasás szegeszkéjével logálja össze --, legalább 15 nappal annak pozása előtt. Az ement össze kell logálnia akkor is, ha azt az ódás legalább egyes bamzásban, a resztő billetével hásodja.
		\cite{Fazekas,Tomacs}

\chapter{Rendszerről összefoglalóan}
	\section{Szakasz címe}
		\subsection{Előnyök és hátrányok}
		Lórum ipse olyan borzasztóan cogális patás, ami fogás nélkül nem varkál megfelelően. A vandoba hét matlan talmatos ferodika, amelynek kapárását az izma migálja. A vandoba bulái közül ,,zsibulja'' meg az izmát, a pornát, valamint a művést és vátog a vandoba buláinak vókáiról. Vókája a raktil prozása két emen között. Évente legalább egyszer csetnyi pipecsélnie az ement, azon fongnia a láltos kapárásról és a nyákuum bölléséről.
		\cite[102.~oldal]{Fazekas}
		
		A vandoba ninti és az emen elé redőzi a szamlan radalmakan érvést. Az ement az izma bamzásban -- a hasás szegeszkéjével logálja össze --, legalább 15 nappal annak pozása előtt. Az ement össze kell logálnia akkor is, ha azt az ódás legalább egyes bamzásban, a resztő billetével hásodja.
		\cite{Fazekas,Tomacs}
	\section{rajzok ismertetése}

\chapter{Rendszerünk működése}
	\section{Terepasztalunk működése}
	\section{optimalizálás}
		Lórum ipse olyan borzasztóan cogális patás, ami fogás nélkül nem varkál megfelelően. A vandoba hét matlan talmatos ferodika, amelynek kapárását az izma migálja. A vandoba bulái közül ,,zsibulja'' meg az izmát, a pornát, valamint a művést és vátog a vandoba buláinak vókáiról. Vókája a raktil prozása két emen között. Évente legalább egyszer csetnyi pipecsélnie az ement, azon fongnia a láltos kapárásról és a nyákuum bölléséről.
		\cite[102.~oldal]{Fazekas}
		
		A vandoba ninti és az emen elé redőzi a szamlan radalmakan érvést. Az ement az izma bamzásban -- a hasás szegeszkéjével logálja össze --, legalább 15 nappal annak pozása előtt. Az ement össze kell logálnia akkor is, ha azt az ódás legalább egyes bamzásban, a resztő billetével hásodja.
		\cite{Fazekas,Tomacs}
		\subsection{telepített napcellák optimalizálása, tájolása}
	\section{Termelők és fogyasztók}
		\subsection{termelők ismertetése}
		\subsection{fogyasztók ismertetése}
		\par A fényforrások, és egyéb elektronikai eszközök a különböző energia felhasználású fogyasztókat fogják modellezni. Célunk valósághűen modellezni a fogyasztókat. (Kisméretű házak, épületek.) 
		\par Az alábbi modellek lesznek a terepasztalunkon:
		\begin{itemize}
			\item Családi házak (átlag 4 fős, fogyasztása körülbelül 230 kWh/hó)
			\item Bérházak (átlag 4 fős, fogyasztása körülbelül 200 kWh/hó)
			\item Tömbházak (bérházak fogyasztásától függően változik)
			\item Elektromos töltőállomások (használattól függően változik)
			\item Közvilágítás (alkalmazástól függően változik)
		\end{itemize}
	
		Modellünk olyan fogyasztási értékeket fog szemléltetni, mely a valóságnak arányosan eleget tesz. A fogyasztók számát dinamikusan lehet majd szabályozni, mely hatással lesz a rendszer működésére. A modellünkben a fogyasztók különböző nyitófeszültségű LED-ek lesznek, melyekkel a fogyasztók energiafelhasználását tudjuk szimbolizálni. A projektünkben a fogyasztók egységes áramot használnak, azonban a számítások során a valóságnak megfelelő értékekkel számolunk.
		
	\section{modellek}
		\subsection{időjárás állomás}
	\section{prototípusok}
		\subsection{telepített napcellák prototípusai}
		\subsection{intelligens napcellák prototípusai}
 	\section{Napcellák}
 		\subsection{telepített napcellák}
 		\subsection{intelligens napcellák}
 		\subsection{napcellák integrációja}
 	\section{Vízerőmű}
 	
 	

	
	
		

\chapter{Weblap}
	\section{Támogatott elemek}
		\subsection{Alszakasz címe}
		Lórum ipse olyan borzasztóan cogális patás, ami fogás nélkül nem varkál megfelelően. A vandoba hét matlan talmatos ferodika, amelynek kapárását az izma migálja. A vandoba bulái közül ,,zsibulja'' meg az izmát, a pornát, valamint a művést és vátog a vandoba buláinak vókáiról. Vókája a raktil prozása két emen között. Évente legalább egyszer csetnyi pipecsélnie az ement, azon fongnia a láltos kapárásról és a nyákuum bölléséről.
		\cite[102.~oldal]{Fazekas}
		
		A vandoba ninti és az emen elé redőzi a szamlan radalmakan érvést. Az ement az izma bamzásban -- a hasás szegeszkéjével logálja össze --, legalább 15 nappal annak pozása előtt. Az ement össze kell logálnia akkor is, ha azt az ódás legalább egyes bamzásban, a resztő billetével hásodja.
		\cite{Fazekas,Tomacs}
	\section{CodeIgniter fejlesztői környezet}
	\section{Adatbázis}

\chapter{Fejlesztői környezetek és publikációi}
	\section{Git verziókövető rendszer}
	\section{Trello feladatkövető rendszer}
	\section{Technológiák}
		\subsection{Python}
		Lórum ipse olyan borzasztóan cogális patás, ami fogás nélkül nem varkál megfelelően. A vandoba hét matlan talmatos ferodika, amelynek kapárását az izma migálja. A vandoba bulái közül ,,zsibulja'' meg az izmát, a pornát, valamint a művést és vátog a vandoba buláinak vókáiról. Vókája a raktil prozása két emen között. Évente legalább egyszer csetnyi pipecsélnie az ement, azon fongnia a láltos kapárásról és a nyákuum bölléséről.
		\cite[102.~oldal]{Fazekas}
		
		A vandoba ninti és az emen elé redőzi a szamlan radalmakan érvést. Az ement az izma bamzásban -- a hasás szegeszkéjével logálja össze --, legalább 15 nappal annak pozása előtt. Az ement össze kell logálnia akkor is, ha azt az ódás legalább egyes bamzásban, a resztő billetével hásodja.
		\cite{Fazekas,Tomacs}
		\subsection{Python és a C nyelv integrációja}
		Lórum ipse olyan borzasztóan cogális patás, ami fogás nélkül nem varkál megfelelően. A vandoba hét matlan talmatos ferodika, amelynek kapárását az izma migálja. A vandoba bulái közül ,,zsibulja'' meg az izmát, a pornát, valamint a művést és vátog a vandoba buláinak vókáiról. Vókája a raktil prozása két emen között. Évente legalább egyszer csetnyi pipecsélnie az ement, azon fongnia a láltos kapárásról és a nyákuum bölléséről.
		\cite[102.~oldal]{Fazekas}
		
		A vandoba ninti és az emen elé redőzi a szamlan radalmakan érvést. Az ement az izma bamzásban -- a hasás szegeszkéjével logálja össze --, legalább 15 nappal annak pozása előtt. Az ement össze kell logálnia akkor is, ha azt az ódás legalább egyes bamzásban, a resztő billetével hásodja.
		\cite{Fazekas,Tomacs}
		\subsection{PHP nyelv}
	\section{Arduino}
		\subsection{szenzorok és kellékek ismertetése}
		


\begin{tetel}
Tétel szövege.
\end{tetel}

\begin{proof}
Bizonyítás szövege.
\end{proof}

\begin{definicio}
Definíció szövege.
\end{definicio}

\begin{megjegyzes}
Megjegyzés szövege.
\end{megjegyzes}

\begin{thebibliography}{2}
\bibitem{Fazekas}
\textsc{Fazekas István}: \emph{Valószínűségszámítás}, Debreceni Egyetem, Debrecen, 2004.
\bibitem{Tomacs}
\textsc{Tómács Tibor}: \emph{A valószínűségszámítás alapjai}, Líceum Kiadó, Eger, 2005.
\end{thebibliography}
\end{document}